\documentclass{article}
\usepackage[letterpaper, textwidth=5.0in, textheight=9.0in]{geometry}

\frenchspacing

\usepackage{amsmath, amsthm, amsfonts}
\let\Re\relax 
\DeclareMathOperator{\Re}{Re}
\DeclareMathOperator{\dd}{d\!}
\newcommand{\mean}[1]{\langle{#1}\rangle}
\newcommand{\given}{\,|\,}
\theoremstyle{definition}
\newcommand{\problemname}{Problem}
\newtheorem{problem}{\problemname}
\newcommand{\probref}[1]{\problemname~\ref{#1}}
\newtheorem{conjecture}{Conjecture}[problem]

\begin{document}
    
\section*{Group averaging of exoplanet likelihoods}
\noindent
\textit{by} \textbf{Hogg} \textit{for} \textbf{Shu} \textit{with help from} \textbf{Villar}.

\paragraph{Abstract:}
The idea is to figure out how to do group averages (or Bayesian mar\-gin\-a\-li\-za\-tions) of likelihood functions, very fast.

\begin{problem}[absolute toy]\label{prob:toy}
Imagine that we have a code handle to a scalar function $f(\theta)\in\mathbb{R}$ that is a function of an angle $\theta\in U(1)$.
We don't know the form or value of this function \textsl{a priori} but we have code that can evaluate the function at any angle $0\leq\theta<2\pi$ or at any list of angles.
We believe that this function could be reasonably well approximated by a low-degree polynomial of trig functions, or
\begin{equation}
    f(\theta) \approx \Re\left\{\sum_{m=0}^M a_m\,\exp(i\,m\,\theta)\right\} ~,
\end{equation}
where $M$ is a small integer, and the $a_m$ are complex numbers (which we don't know).
We seek to find an estimate or value of the average $\mean{f}$ or marginalization
\begin{equation}
    \mean{f} = \frac{1}{2\pi}\,\int_0^{2\pi} f(\theta)\,\dd\theta
\end{equation}
taking as few evaluations of $f(\theta)$ as possible.
Additionally, we would like some assessment of the quality of the approximation, or at least a flag if the approximation is poor.
\end{problem}

\begin{conjecture}
If $f(\theta)$ is indeed well approximated by a polynomial of degree $M$, then we can get a good estimate of the average $\mean{f}$ with only $2\,M+1$ function evaluations.
\end{conjecture}

\begin{conjecture}
If you want an estimate of $\mean{f}$ and \emph{also} some kind of confidence or check of the integral, you will have to do substantially more than $2\,M+1$ function evaluations.
\end{conjecture}

\begin{problem}[exponentiated toy]\label{prob:exp}
Same as \probref{prob:toy}---we want to know $\mean{f}$---but now it is the logarithm of the function that is well described by the polynomial; that is
\begin{equation}
    \ln f(\theta) \approx \Re\left\{\sum_{m=0}^M a_m\,\exp(i\,m\,\theta)\right\} ~.
\end{equation}
\end{problem}

\begin{conjecture}
The solution will also only take $2\,M+1$ function evaluations, but, subsequently, a numerical integration will be required.
That is, there won't be any simple form for the integral, given the function evaluations.
\end{conjecture}

\begin{problem}[toy likelihood function]\label{prob:toylike}
Choose a random (latent) non-zero 2-vector $z\in\mathbb{R}^2$.
Create some ``data'' $y$ to be a noisy estimate of this 2-vector, such that
\begin{equation}
    y = z + \epsilon ~,
\end{equation}
and $\epsilon$ is drawn from a 2-dimensional normal with zero mean and variance tensor $V=\sigma^2\,I_2$, where $I_2$ is the $2\times 2$ identity matrix.
Now imagine that we are trying to estimate $z$, given the data $y$, and knowing $\sigma$.
Write code to compute the likelihood function $p(y\given z)$.

Since we can parameterize the vector $z$ in terms of a magnitude $r$ and an angle $\theta$ (with respect to a coordinate axis, say), this likelihood function can be written as $p(y\given r,\theta)$, which is a function of $\theta$.
Find the average of this function using the techniques developed in \probref{prob:toy} and \probref{prob:exp}.
What issues do you encounter and how do you find $M$ values that are reasonable?
How should you plot or present your results? Treat the data $y$ to be fixed, but the magnitude $r$ to be variable.
\end{problem}

\begin{conjecture}
The value of $M$ will probably depend on whether you are using your solution to \probref{prob:toy} or to \probref{prob:exp}.
It will also depend on the ratio $|z|/\sigma$ of the magnitude $|z|$ of the true $z$ you chose at the beginning to the uncertainty root-variance $\sigma$.
This quantity is what an astronomer might call ``the signal-to-noise ratio.''
\end{conjecture}

\end{document}
